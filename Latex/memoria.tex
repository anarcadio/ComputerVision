\documentclass[a4paper,openright, 12pt]{book}
\usepackage[spanish]{babel} 
\usepackage[utf8]{inputenc} 
\usepackage{listings}
\usepackage{color}

\definecolor{mygreen}{rgb}{0,0.6,0}
\definecolor{mygray}{rgb}{0.5,0.5,0.5}
\definecolor{mymauve}{rgb}{0.58,0,0.82}

\lstset{ %
  backgroundcolor=\color{white},   
  basicstyle=\footnotesize,       
  breakatwhitespace=false,         
  breaklines=true,                 % sets automatic line breaking
  captionpos=b,                    % sets the caption-position to bottom
  commentstyle=\color{mygreen},    % comment style
  deletekeywords={...},            % if you want to delete keywords from the given language
  escapeinside={\%*}{*)},          % if you want to add LaTeX within your code
  extendedchars=true,              % lets you use non-ASCII characters; for 8-bits encodings only, does not work with UTF-8
  frame=single,                    % adds a frame around the code
  keepspaces=true,                 % keeps spaces in text, useful for keeping indentation of code (possibly needs columns=flexible)
  keywordstyle=\color{blue},       % keyword style
  language=Python,                 % the language of the code
  morekeywords={*,...},            % if you want to add more keywords to the set
  numbers=none,                    % where to put the line-numbers; possible values are (none, left, right)
  numbersep=5pt,                   % how far the line-numbers are from the code
  numberstyle=\tiny\color{mygray}, % the style that is used for the line-numbers
  rulecolor=\color{black},         % if not set, the frame-color may be changed on line-breaks within not-black text (e.g. comments (green here))
  showspaces=false,                % show spaces everywhere adding particular underscores; it overrides 'showstringspaces'
  showstringspaces=false,          % underline spaces within strings only
  showtabs=false,                  % show tabs within strings adding particular underscores
  stepnumber=2,                    % the step between two line-numbers. If it's 1, each line will be numbered
  stringstyle=\color{mymauve},     % string literal style
  tabsize=2,                       % sets default tabsize to 2 spaces
  title=\lstname                   % show the filename of files included with \lstinputlisting; also try caption instead of title
}




\begin{document}

\begin{titlepage}
\begin{center}
\begin{Huge}
\textsc{Título}
\end{Huge}
\end{center}
\end{titlepage}

\newpage
\mbox{}
\thispagestyle{empty} 


\tableofcontents % indice de contenidos
\thispagestyle{empty}


\chapter{Introducción}\label{cap.introduccion}
\pagenumbering{arabic} % para empezar la numeración con números

\section{Motivación}


\section{OpenCV}

OpenCV es una librería de código abierto escrita en C y C++ destinada a la visión artificial y al tratamiento de imágenes. Se trata de una librería multiplataforma con versiones para GNU/Linux, Windows, Mac OS y Android y actualmente cuenta con interfaces para Python, Java y MATLAB/OCTAVE. Las últimas versiones incluyen soporte para GPU usando CUDA.
Desarrollada originalmente por Intel, su primera versión alfa se publicó en el año 2000 en el \textit{IEEE Conference on Computer Vision and Pattern Recognition}. OpenCV nació inicialmente como un proyecto para avanzar en las aplicaciones de uso intenso de la CPU y dando gran importancia a las aplicaciones en tiempo real. Hoy en día cuenta con más 2500 algoritmos optimizados que abarcan todo tipo de campos relacionados con la visión artificial.
Estos algortimos pueden ser usados para tareas como: detectar y reconocer caras y gestos, identificar objetos, detección de características 2D y 3D, estimación de movimiento, seguimiento del movimiento, visión estéro y calibración de la cámara, eliminar los ojos rojos de las fotografías realizadas con flash...
\newline
OpenCV es ampliamente utilizada por todo tipo de empresas (desde grandes empresas como Google, Yahoo, Microsoft, Intel, IBM, Sony, Honda, Toyota a pequeñas empresas), grupos de investigación y organismos gubernamentales y en sectores de todo tipo como: inspección de los productos en las fábricas, seguridad, usos médicos, robótica...
\newline
\subsection*{Ejemplos}
\lstinputlisting[language=Python, frame= single]{ejemplos/ejemplo1.py}
\lstinputlisting[language=Python, frame= single]{ejemplos/ejemplo2.py}
\lstinputlisting[language=Python, frame= single]{ejemplos/ejemplo3.py}
\lstinputlisting[language=Python, frame= single]{ejemplos/ejemplo4.py}
\lstinputlisting[language=Python, frame= single]{ejemplos/ejemplo5.py}
\lstinputlisting[language=Python, frame= single]{ejemplos/ejemplo7.py}
\lstinputlisting[language=Python, frame= single]{ejemplos/ejemplo9.py}
\lstinputlisting[language=Python, frame= single]{ejemplos/BlurCam.py}

\newpage
\section{Android}

Android es un sistema operativo desarrollado por Google y orientado a móviles y tablets. Está basado en linux y es de código abierto. Actualmente se estima que en torno un 80\% de los dispositívos móviles usan el sistema operativo android.
Su núcleo está programado en C, pero la aplicaciones y toda la interfaz de usuario se programan en Java.
Este sistema operativo está estructurado en 4 capas:
\begin{description}
  \item[Núcleo linux] \hfill \\
  Se encarga de las funcionalidades básicas del sistema, como manejo de procesos, de la memoria y de los dispositivos como la cámara, la pantalla, etc.
   Asimismo funciona como una capa de ebstracción entre el hardware y el resto del software.
   
  \item[Librerías y Runtime de Android] \hfill \\
  Por encima del núcleo, hay una serie de librerías usadas por componentes del sistema, entre ellas destacan Surface Manager, Media Framework, SQLite, WebKit, SGL y Open GL
  \newline
  El runtime de Android proporciona un componente clave,la máquina virtual Dalvik. Cada aplicación Android corre su propio proceso, con su propia instancia de esta máquina virtual.
  Además, el runtime de android proporciona librerías básicas que proporcionan la mayor parte de las funciones del lenguaje de programación Java.
  
  \item[Framework de aplicaciones] \hfill \\
  Esta capa proporciona a las aplicaciones muchos servicios en forma de clases de Java 
  
   \item[Aplicaciones] \hfill \\
  En esta capa se encuentran tanto las aplicaciones base como las que instalemos y desarrollemos   \ldots
\end{description}

\subsection*{Estructura de una aplicación Android}
Las aplicaciones android están por los llamados componentes de aplicación. Hay cuatro tipo de componentes:
\begin{description}
  \item[Activities] \hfill \\
  Representa una única pantalla de la aplicación con su interfaz de usuario.
  \item[Services] \hfill \\
  Son componentes que se ejecutan en segundo plano y que no constan de interfaz gráfica.
  \item[Content providers] \hfill \\
  \item[Broadcast receivers] \hfill \\
\end{description}

\chapter{Estabilización de imagen}\label{cap.}

\section{Algoritmo}
Se trata de un algoritmo básico de estabilización de imagen que se basa en la suposición de que el único movimiento de la cámara es el que queremos eliminar; es decir, excepto por ligeros movimientos no deseados, la cámara está estática.
La idea del algoritmo es muy sencilla: 
Tomamos dos frames consecutivos y buscamos una serie de puntos característicos mediante el algoritmo propuesto por Shi y Tomashi \cite{shiandtomasi}, usando la función: \lstinline{goodFeaturesToTrack}; después vemos a donde se han movido esos puntos en el siguiente frame mediante el algoritmo de flujo óptico de Lucas-Kanade.
Finalmente calculamos la homografía que lleva los puntos originales a donde hemos calculado que se han movido y le aplicamos la inversa de esa transformación al segundo frame para colocarlo donde debería estar.
\subsection{Encontrar puntos característicos}

EXPLICACIÓN
\subsection{Algoritmo de Lucas-Kanade}
EXPLICACIÓN

\subsection{Codigo}
\lstinputlisting[language=Python, frame= single]{estabilizacion.py}

\section{Uso del acelerometro para estabilizar}
Se trata de intentar utilizar los datos extra de los que disponemos, es decir, los que nos proporciona el acelerómetro del móvil para intentar estabilizar la imagen utilizando estos datos.
Para ello en primer lugar hacemos un programa para android que se encarga de recoger estos datos, grabando el vídeo a la vez que registra los datos del acelerómetro y los guarda en un archivo junto con el momento exacto en el que se han registrado los datos.
Luego, un segundo programa en el ordenador proceaso los arhivos para intentar estabilizar la imagen: recoge los datos de aceleración en cada instante de media y aproxima el movimiento en dichos instantes. Luego calcula mediante interpolación cuanto se ha movido la camára en el instante del fotograma y recoloca el fotograma de acuerdo con lo obtenido.

\subsection{Programa para Android}


\subsection{Programa de procesado de los datos}
\lstinputlisting[language=Python, frame= single]{estabilizacion_acelerometro.py}

\subsection{Resultados}

\cleardoublepage
\addcontentsline{toc}{chapter}{Bibliografía}
\bibliographystyle{plain} % estilo de la bibliografía.
\bibliography{texto} % texto.bib es el fichero donde está salvada la bibliografía.


\end{document}
